\documentclass{article}

\usepackage{fancyhdr}
\usepackage{extramarks}
\usepackage{amsmath}
\usepackage{amsthm}
\usepackage{amsfonts}
\usepackage{tikz}
\usepackage[plain]{algorithm}
\usepackage{algpseudocode}
\usepackage{listings}
\lstset{breaklines=true}

\usetikzlibrary{automata,positioning}

%
% Basic Document Settings
%

\topmargin=-0.45in
\evensidemargin=0in
\oddsidemargin=0in
\textwidth=6.5in
\textheight=9.0in
\headsep=0.25in

\linespread{1.1}

\pagestyle{fancy}
\lhead{\hmwkAuthorName}
\chead{\hmwkClass\ (\hmwkClassInstructor\ \hmwkClassTime): \hmwkTitle}
\rhead{\firstxmark}
\lfoot{\lastxmark}
\cfoot{\thepage}

\renewcommand\headrulewidth{0.4pt}
\renewcommand\footrulewidth{0.4pt}

\setlength\parindent{0pt}

%
% Create Problem Sections
%

\newcommand{\enterProblemHeader}[1]{
    \nobreak\extramarks{}{Problem \arabic{#1} continued on next page\ldots}\nobreak{}
    \nobreak\extramarks{Problem \arabic{#1} (continued)}{Problem \arabic{#1} continued on next page\ldots}\nobreak{}
}

\newcommand{\exitProblemHeader}[1]{
    \nobreak\extramarks{Problem \arabic{#1} (continued)}{Problem \arabic{#1} continued on next page\ldots}\nobreak{}
    \stepcounter{#1}
    \nobreak\extramarks{Problem \arabic{#1}}{}\nobreak{}
}

\setcounter{secnumdepth}{0}
\newcounter{partCounter}
\newcounter{homeworkProblemCounter}
\setcounter{homeworkProblemCounter}{1}
\nobreak\extramarks{Problem \arabic{homeworkProblemCounter}}{}\nobreak{}

%
% Homework Problem Environment
%
% This environment takes an optional argument. When given, it will adjust the
% problem counter. This is useful for when the problems given for your
% assignment aren't sequential. See the last 3 problems of this template for an
% example.
%
\newenvironment{homeworkProblem}[1][-1]{
    \ifnum#1>0
        \setcounter{homeworkProblemCounter}{#1}
    \fi
    \section{Problem \arabic{homeworkProblemCounter}}
    \setcounter{partCounter}{1}
    \enterProblemHeader{homeworkProblemCounter}
}{
    \exitProblemHeader{homeworkProblemCounter}
}

%
% Homework Details
%   - Title
%   - Due date
%   - Class
%   - Section/Time
%   - Instructor
%   - Author
%

\newcommand{\hmwkTitle}{Homework\ \#1}
\newcommand{\hmwkDueDate}{January 9, 2015}
\newcommand{\hmwkClass}{Graph Theory}
\newcommand{\hmwkClassTime}{MWF 9:15}
\newcommand{\hmwkClassInstructor}{Professor McGinley}
\newcommand{\hmwkAuthorName}{Rick Sullivan}

%
% Title Page
%

\title{
    \vspace{2in}
    \textmd{\textbf{\hmwkClass:\ \hmwkTitle}}\\
    \normalsize\vspace{0.1in}\small{Due\ on\ \hmwkDueDate}\\
    \vspace{0.1in}\large{\textit{\hmwkClassInstructor\ \hmwkClassTime}}
    \vspace{3in}
}

\author{\textbf{\hmwkAuthorName}}
\date{}

\renewcommand{\part}[1]{\textbf{\large Part \Alph{partCounter}}\stepcounter{partCounter}\\}

%
% Various Helper Commands
%

% Useful for algorithms
\newcommand{\alg}[1]{\textsc{\bfseries \footnotesize #1}}

% For derivatives
\newcommand{\deriv}[1]{\frac{\mathrm{d}}{\mathrm{d}x} (#1)}

% For partial derivatives
\newcommand{\pderiv}[2]{\frac{\partial}{\partial #1} (#2)}

% Integral dx
\newcommand{\dx}{\mathrm{d}x}

% Alias for the Solution section header
\newcommand{\solution}{\textbf{\large Solution}}

% Probability commands: Expectation, Variance, Covariance, Bias
\newcommand{\E}{\mathrm{E}}
\newcommand{\Var}{\mathrm{Var}}
\newcommand{\Cov}{\mathrm{Cov}}
\newcommand{\Bias}{\mathrm{Bias}}

\begin{document}

\maketitle

\pagebreak

\begin{homeworkProblem}
    Prove or disprove: if every vertex of a simple graph G has degree two, the G is a cycle.
    \\

    \textbf{Solution}
    \\

    Consider the following graph:
    \\

    \tikzset{main node/.style={circle,draw,minimum size=0.5cm,inner sep=0pt},
            }
    \begin{tikzpicture}
        \node[main node] (1) {$1$};
        \node[main node] (2) [below left = 1.3cm and 1.0cm of 1]  {$2$};
        \node[main node] (3) [below right = 1.3cm and 1.0cm of 1] {$3$};
        \node[main node] (4) [right = 4cm of 1]                  {$4$};
        \node[main node] (5) [below left = 1.3cm and 1.0cm of 4]  {$5$};
        \node[main node] (6) [below right = 1.3cm and 1.0cm of 4] {$6$};

        \path[draw,thick]
        (1) edge node {} (2)
        (2) edge node {} (3)
        (3) edge node {} (1)
        (4) edge node {} (5)
        (5) edge node {} (6)
        (6) edge node {} (4);
    \end{tikzpicture}

    Each node has degree two and the graph has cycles contained within it, but the graph as a whole is not a cycle.
\end{homeworkProblem}

\begin{homeworkProblem}
    Determine the maximum size of a clique and the maximum size of an indepenedent set of the graph below.
    \\

    \tikzset{main node/.style={circle,draw,minimum size=0.5cm,inner sep=0pt},
            }
    \begin{tikzpicture}
        \node[main node] (1) {$1$};
        \node[main node] (2) [below left = 1.3cm and 2.0cm of 1]  {$2$};
        \node[main node] (3) [below left = 1.3cm and 0.5cm of 1] {$3$};
        \node[main node] (4) [below right = 1.3cm and 0.5cm of 1] {$4$};
        \node[main node] (5) [below right = 1.3cm and 2.0cm of 1]  {$5$};
        \node[main node] (6) [below = 2.6cm of 1] {$6$};

        \path[draw,thick]
        (1) edge node {} (2)
        (1) edge node {} (3)
        (1) edge node {} (4)
        (1) edge node {} (5)
        (2) edge node {} (3)
        (3) edge node {} (4)
        (4) edge node {} (5)
        (6) edge node {} (2)
        (6) edge node {} (3)
        (6) edge node {} (4)
        (6) edge node {} (5)
        (1) edge node [bend right] {} (6);
    \end{tikzpicture}
    \\

    \textbf{Solution}
    \\

    The maximum clique size is 4, consisting of nodes 1, 6 and any other two adjacent nodes from the middle row.

    The maximum size of an independent set is only 2, either the set \{2, 4\} or the set \{3, 5\}.

\end{homeworkProblem}

\begin{homeworkProblem}
    Determine which pairs of graphs below are isomorphic.
    \\

    \usetikzlibrary{shapes.geometric}
    \begin{tikzpicture}
        % create the node
        \node[draw, minimum size=2cm,regular polygon,regular polygon sides=7] (a) {A};
        \node[right=1cm of a, draw, minimum size=2cm,regular polygon,regular polygon sides=7] (b) {B};
        \node[right=1cm of b, draw,minimum size=2cm,regular polygon,regular polygon sides=7] (c) {C};
        \node[right=1cm of c, draw,minimum size=2cm,regular polygon,regular polygon sides=6] (d) {D};
        \node[right=1cm of d, draw,minimum size=2cm,regular polygon,regular polygon sides=7] (e) {E};

        % draw a black dot in each vertex
        \foreach \x in {1,2,...,7}
            \foreach \a in {a, b, c, e}{
              \fill (\a.corner \x) circle[radius=2pt];
              \node[right]at(\a.corner \x) (\a\x) {\a\x};
            }

        \foreach \x in {1,2,...,6}{
          \fill (d.corner \x) circle[radius=2pt];
          \node[right]at(d.corner \x) (d\x) {d\x};
          }
        \fill (d.center) circle[radius=2pt];
          \node[right]at(d.center) (d7) {d7};
    \end{tikzpicture}
    \\

    \textbf{Solution}
    \\

    Graphs A, B, and E are all isomorphic, as are graphs C and D.

    The isomorphisms can be constructed using the following mappings:
    \\

    \begin{tabular}{l | l}
        A & B\\
        \hline
        a1 & b1\\
        a2 & b6\\
        a3 & b7\\
        a4 & b5\\
        a5 & b4\\
        a6 & b3\\
        a7 & b2\\
    \end{tabular}
    \hspace{0.5in}
    \begin{tabular}{l | l}
        A & E\\
        \hline
        a1 & e1\\
        a2 & e4\\
        a3 & e7\\
        a4 & e3\\
        a5 & e6\\
        a6 & e2\\
        a7 & e5\\
    \end{tabular}
    \hspace{0.5in}
    \begin{tabular}{l | l}
        C & D\\
        \hline
        c1 & d3\\
        c2 & d5\\
        c3 & d6\\
        c4 & d2\\
        c5 & d1\\
        c6 & d7\\
        c7 & d4\\
    \end{tabular}
\end{homeworkProblem}

\begin{homeworkProblem}
    Prove that \(G_1 \times G_2\) is isomorphic to \(G_2 \times G_1\).
    \\

    \textbf{Solution}
    \begin{proof}
        Consider two graphs, \(G_1\) and \(G_2\), with vertices \(v_{a1}, v_{a2}, ..., v_{an}\) and \(v_{b1}, v_{b2}, ..., v_{bn}\), and edges \(e_{a1}, e_{a2}, ..., e_{an}\) and \(e_{b1}, e_{b2}, ..., e_{bn}\), respectively.
        \\

        The product \(G_1 \times G_2\) will have vertices 
        \((v_{a1}, v_{b1}), (v_{a1}, v_{b2}), ..., (v_{an}, v_{bn}), (v_{b1}, v_{a1}), ... (v_{bn}, v_{an})\). 
        Edges will exist between any two vertices \((v_1, w_1)\) and \((v_2, w_2)\) if and only if either \(v_1 = v_2\) and \(w_1\) and \(w_2\) shared an edge in \(G_2\), or vice versa.
        \\

        The product \(G_2 \times G_1\) follows a symmetrical pattern. In this case, the vertices of the product will be 
        \((v_{b1}, v_{a1}), (v_{b1}, v_{a2}), ..., (v_{bn}, v_{an}), (v_{a1}, v_{b1}), ... (v_{an}, v_{bn})\). 
        Edges will similarly be symmetrical.
        \\

        We can produce an isomorphism by mapping vertex pairs in one cartesian product to the same vertex in the other. For example, the vertex \((v_{a1}, v_{b1})\) in \(G_1 \times G_2\) will actually map to \((v_{a1}, v_{b1})\) in \(G_2 \times G_1\).
    \end{proof}
\end{homeworkProblem}

\begin{homeworkProblem}
    Prove that a \(p, q\) graph is complete if and only if \(q = {p \choose 2}\).
    \\

    \textbf{Proving completeness leads to \({p \choose 2}\) edges}
    \begin{proof}
    Assume that a graph G with p vertices is complete. By definition of graph completeness, all possible edges will exist. Edges, in a simple undirected graph, can be represented as a set of two vertices that make up the edge's endpoints. Therefore, the total number of possible edges is the total possible number of two-vertex sets from a vertex set of size p, or \({p \choose 2}\). Therefore, complete graph G must have \({p \choose 2}\) edges.
    \end{proof}

    \textbf{Proving \({p \choose 2}\) edges leads to completeness}
    \begin{proof}
    Assume that a graph G with p vertices has \(q = {p \choose 2}\) edges. The maximum possible number of edges in a graph of order p, as explained above, is \(q = {p \choose 2}\). Therefore, every possible edge in G exists. By definition, G is complete.
    \end{proof}

\end{homeworkProblem}

\pagebreak

\end{document}
