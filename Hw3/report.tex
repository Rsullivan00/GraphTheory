\documentclass{article}

\usepackage{fancyhdr}
\usepackage{extramarks}
\usepackage{amsmath}
\usepackage{amsthm}
\usepackage{amsfonts}
\usepackage{tikz}
\usepackage{tikz-qtree}
\usepackage[plain]{algorithm}
\usepackage{algpseudocode}
\usepackage{listings}
\usepackage{enumerate}
\lstset{breaklines=true}

\usetikzlibrary{automata,positioning}

%
% Basic Document Settings
%

\topmargin=-0.45in
\evensidemargin=0in
\oddsidemargin=0in
\textwidth=6.5in
\textheight=9.0in
\headsep=0.25in

\linespread{1.1}

\pagestyle{fancy}
\lhead{\hmwkAuthorName}
\chead{\hmwkClass\ (\hmwkClassInstructor): \hmwkTitle}
\rhead{\firstxmark}
\lfoot{\lastxmark}
\cfoot{\thepage}

\renewcommand\headrulewidth{0.4pt}
\renewcommand\footrulewidth{0.4pt}

\setlength\parindent{0pt}

%
% Create Problem Sections
%

\newcommand{\enterProblemHeader}[1]{
    \nobreak\extramarks{}{Problem \arabic{#1} continued on next page\ldots}\nobreak{}
    \nobreak\extramarks{Problem \arabic{#1} (continued)}{Problem \arabic{#1} continued on next page\ldots}\nobreak{}
}

\newcommand{\exitProblemHeader}[1]{
    \nobreak\extramarks{Problem \arabic{#1} (continued)}{Problem \arabic{#1} continued on next page\ldots}\nobreak{}
    \stepcounter{#1}
    \nobreak\extramarks{Problem \arabic{#1}}{}\nobreak{}
}

\setcounter{secnumdepth}{0}
\newcounter{partCounter}
\newcounter{homeworkProblemCounter}
\setcounter{homeworkProblemCounter}{1}
\nobreak\extramarks{Problem \arabic{homeworkProblemCounter}}{}\nobreak{}

%
% Homework Problem Environment
%
% This environment takes an optional argument. When given, it will adjust the
% problem counter. This is useful for when the problems given for your
% assignment aren't sequential. See the last 3 problems of this template for an
% example.
%
\newenvironment{homeworkProblem}[1][-1]{
    \ifnum#1>0
        \setcounter{homeworkProblemCounter}{#1}
    \fi
    \section{Problem \arabic{homeworkProblemCounter}}
    \setcounter{partCounter}{1}
    \enterProblemHeader{homeworkProblemCounter}
}{
    \exitProblemHeader{homeworkProblemCounter}
}

%
% Homework Details
%   - Title
%   - Due date
%   - Class
%   - Section/Time
%   - Instructor
%   - Author
%

\newcommand{\hmwkTitle}{Homework\ \#3}
\newcommand{\hmwkDueDate}{January 23, 2015}
\newcommand{\hmwkClass}{Graph Theory}
\newcommand{\hmwkClassTime}{MWF 9:15}
\newcommand{\hmwkClassInstructor}{Professor McGinley}
\newcommand{\hmwkAuthorName}{Rick Sullivan}

%
% Title Page
%

\title{
    \vspace{2in}
    \textmd{\textbf{\hmwkClass:\ \hmwkTitle}}\\
    \normalsize\vspace{0.1in}\small{Due\ on\ \hmwkDueDate}\\
    \vspace{0.1in}\large{\textit{\hmwkClassInstructor\ \hmwkClassTime}}
    \vspace{3in}
}

\author{\textbf{\hmwkAuthorName}}
\date{}

\renewcommand{\part}[1]{\textbf{\large Part \Alph{partCounter}}\stepcounter{partCounter}\\}

%
% Various Helper Commands
%

% Useful for algorithms
\newcommand{\alg}[1]{\textsc{\bfseries \footnotesize #1}}

% For derivatives
\newcommand{\deriv}[1]{\frac{\mathrm{d}}{\mathrm{d}x} (#1)}

% For partial derivatives
\newcommand{\pderiv}[2]{\frac{\partial}{\partial #1} (#2)}

% Integral dx
\newcommand{\dx}{\mathrm{d}x}

% Alias for the Solution section header
\newcommand{\solution}{\textbf{\large Solution}}

% Probability commands: Expectation, Variance, Covariance, Bias
\newcommand{\E}{\mathrm{E}}
\newcommand{\Var}{\mathrm{Var}}
\newcommand{\Cov}{\mathrm{Cov}}
\newcommand{\Bias}{\mathrm{Bias}}

\begin{document}

\maketitle

\pagebreak

\begin{homeworkProblem}
    Give \(k(G)\) and \(k_1(G)\) for the following graphs:

    \tikzset{main node/.style={circle,draw,minimum size=0.25cm,inner sep=0pt},}
    \begin{tikzpicture}
        \node[main node] (a) {};
        \node[main node] (b) [below left = 0.6cm and 0.6cm of a] {};
        \node[main node] (c) [below right = 0.6cm and 0.6cm of a] {};
        \node[main node] (d) [below left = 0.6cm and 0.6cm of c] {};
        \node[main node] (e) [right= 1cm of c] {};
        \node[main node] (f) [below right = 0.6cm and 0.6cm of e] {};
        \node[main node] (g) [above right = 0.6cm and 0.6cm of e] {};

        \path[draw, thick]
        (a) edge node {} (b)
        (a) edge node {} (c)
        (c) edge node {} (d)
        (c) edge node {} (e)
        (e) edge node {} (f)
        (e) edge node {} (g);
    \end{tikzpicture}
    \hspace{0.5in}
    \begin{tikzpicture}
        \node[main node] (a) {};
        \node[main node] (b) [right= 1cm of a] {};
        \node[main node] (c) [right= 1cm of b] {};
        \node[main node] (d) [below= 1cm of a] {};
        \node[main node] (e) [below= 1cm of b] {};
        \node[main node] (f) [below= 1cm of c] {};

        \path[draw, thick]
        (a) edge node {} (b)
        (a) edge [bend left] node {} (d)
        (a) edge node {} (e)
        (b) edge node {} (c)
        (b) edge node {} (e)
        (b) edge node {} (f)
        (c) edge [bend left] node {} (f)
        (d) edge [bend left] node {} (a)
        (d) edge node {} (e)
        (e) edge node {} (f)
        (f) edge [bend left] node {} (c);
    \end{tikzpicture}
    \\
 

    \textbf{Solution}
        
    The first graph has \(k(G) = 1\) and \(k_1(G)=1\). These can be found by removing any vertex with degree > 1, or removing any edge.
    \\

    The second graph has \(k(G) = 2\) and \(k_1(G)=3\). \(k(G)\) requires removing the two central vertices, while \(k_1(G)\) requires removing either set of three edges like so:
    \\

    \begin{tikzpicture}
        \node[main node] (a) {};
        \node[main node] (b) [right= 1cm of a] {};
        \node[main node] (c) [right= 1cm of b] {};
        \node[main node] (d) [below= 1cm of a] {};
        \node[main node] (e) [below= 1cm of b] {};
        \node[main node] (f) [below= 1cm of c] {};

        \path[draw, thick]
        (a) edge [bend left] node {} (d)
        (b) edge node {} (c)
        (b) edge node {} (e)
        (b) edge node {} (f)
        (c) edge [bend left] node {} (f)
        (d) edge [bend left] node {} (a)
        (e) edge node {} (f)
        (f) edge [bend left] node {} (c);
    \end{tikzpicture}
    \hspace{0.25in}
    or
    \hspace{0.25in}
     \begin{tikzpicture}
        \node[main node] (a) {};
        \node[main node] (b) [right= 1cm of a] {};
        \node[main node] (c) [right= 1cm of b] {};
        \node[main node] (d) [below= 1cm of a] {};
        \node[main node] (e) [below= 1cm of b] {};
        \node[main node] (f) [below= 1cm of c] {};

        \path[draw, thick]
        (a) edge node {} (b)
        (a) edge [bend left] node {} (d)
        (a) edge node {} (e)
        (b) edge node {} (e)
        (c) edge [bend left] node {} (f)
        (d) edge [bend left] node {} (a)
        (d) edge node {} (e)
        (f) edge [bend left] node {} (c);
    \end{tikzpicture}
\end{homeworkProblem}

\begin{homeworkProblem}
    For the graph below:
    \begin{enumerate}[a)]
        \item Give three edges whose removal separates \(s\) from \(t\). Is this the minimum number of such edges?
        \item What is the maximum number of edge-disjoint paths from \(s\) to \(t\) and why?
    \end{enumerate}

    \tikzset{main node/.style={circle,draw,minimum size=0.25cm,inner sep=1pt},}
    \begin{tikzpicture}
        \node[main node] (a) {a};
        \node[main node] (b) [below= 1cm of a] {b};
        \node[main node] (c) [below= 1cm of b] {c};
        \node[main node] (d) [right= 1cm of a] {d};
        \node[main node] (s) [left= 1cm of b] {s};
        \node[main node] (e) [below= 1cm of d] {e};
        \node[main node] (f) [below= 1cm of e] {f};
        \node[main node] (t) [right= 1cm of e] {t};

        \path[draw, thick]
        (s) edge node {} (a)
        (s) edge node {} (b)
        (s) edge node {} (c)
        (a) edge node {} (e)
        (b) edge node {} (d)
        (b) edge node {} (f)
        (c) edge node {} (e)
        (d) edge node {} (t)
        (e) edge node {} (t)
        (f) edge node {} (t);
    \end{tikzpicture}
    \\


    \textbf{Part a}
        
    The edge set \({s-a, s-b, s-c}\) separates \(s\) from \(t\) when removed. This is not the minimum number of such edges; the set \({s-b, e-t}\) does the same.
    \\

    \textbf{Part b}

    The maximum number of edge-disjoint paths from \(s\) to \(t\) is 2: every path must take either edge \(s-b\) or \(e-t\). This is confirmed by the second answer of part a. Because the graph can be separated by removing those two edges, one of them must be traversed to travel from one subgraph to the other. 
\end{homeworkProblem}

\begin{homeworkProblem}
    Find the shortest path from \(a\) to the other vertices.

    \tikzset{main node/.style={circle,draw,minimum size=0.5cm,inner sep=1pt}}
    \begin{tikzpicture}[auto, node distance = 2cm]
        \node[main node] (a) {a};
        \node[main node] (b) [right of=a] {b};
        \node[main node] (c) [below left of=a] {c};
        \node[main node] (d) [below right of=b] {d};
        \node[main node] (e) [below right of=c] {e};
        \node[main node] (f) [right of=e] {f};

        \path[draw, thick, ->]
        (c) edge node {5} (a)
        (a) edge node {12} (b)
        (a) edge node {6} (f)
        (a) edge node {1} (e)
        (e) edge node {9} (c)
        (e) edge [bend right] node {9} (f)
        (f) edge node {9} (e)
        (f) edge node {6} (b)
        (f) edge [bend right] node {12} (d)
        (b) edge node {6} (d)
        (d) edge node {10} (f)
        (e) edge node {9} (c);
    \end{tikzpicture}
    \\

    \textbf{Solution}
    \\

    Using Djikstra's algorithm:

    \begin{center}
        \begin{tabular}{l l l l l l l}
            & a & b & c & d & e & f\\
            Initializing: & 0 & \(\infty\) & \(\infty\) & \(\infty\) & \(\infty\) & \(\infty\)\\
            At a: & 0 & 12 & \(\infty\) & \(\infty\) & 1 & 6 \\
            At e: & 0 & 12 & 10 & 18 & 1 & 6 \\
            At f (no change): & 0 & 12 & 10 & 18 & 1 & 6 \\
            At e (no change): & 0 & 12 & 10 & 18 & 1 & 6 \\
            At b (no change): & 0 & 12 & 10 & 18 & 1 & 6 \\
            At d (no change): & 0 & 12 & 10 & 18 & 1 & 6 \\
        \end{tabular}
    \end{center}
\end{homeworkProblem}

\begin{homeworkProblem}
    Find the articulation points (using the algorithm!)

    \tikzset{main node/.style={circle,draw,minimum size=0.25cm,inner sep=1pt}}
    \begin{tikzpicture}
        \node[main node] (f) {F};
        \node[main node] (d) [right= 1cm of f] {D};
        \node[main node] (g) [below= 1cm of d] {G};
        \node[main node] (b) [right= 1cm of d] {B};
        \node[main node] (a) [right= 1cm of b] {A};
        \node[main node] (c) [right= 1cm of a] {C};
        \node[main node] (i) [right= 1cm of c] {I};
        \node[main node] (j) [above right= 1cm of i] {J};
        \node[main node] (e) [below right= 1cm of i] {E};
        \node[main node] (h) [below right= 1cm of j] {H};

        \path[draw, thick]
        (f) edge node {} (d)
        (d) edge node {} (g)
        (d) edge node {} (b)
        (g) edge node {} (b)
        (b) edge node {} (a)
        (a) edge node {} (c)
        (a) edge [bend right] node {} (i)
        (c) edge node {} (i)
        (i) edge node {} (j)
        (i) edge node {} (e)
        (e) edge node {} (h)
        (j) edge node {} (h);
    \end{tikzpicture}
    \\

    \textbf{Solution}

    My constructed DFS tree:\\
    \begin{center}
    \begin{tikzpicture}
        \Tree [.F [.\node(D){D}; [.G [.\node(B){B}; [.\node(A){A}; [.C [.\node(I){I}; [.J [.H [.\node(E){E}; ] ] ] ] ] ] ] ] ] ] 

        \begin{scope}[dashed]
            \draw (B)..controls +(north west:1) and +(south west:1)..(D);
            \draw (I)..controls +(north east:1) and +(south east:1)..(A);
            \draw (E)..controls +(north west:1) and +(south west:1)..(I);
        \end{scope}
    \end{tikzpicture}
    \end{center}

    Initializing our n and c values to the order a node was encountered on our search.\\
    \begin{center}
    \begin{tabular}{c | c c }
        & n & c\\
        \hline
        A & 5 & 5\\
        B & 4 & 4\\
        C & 6 & 6\\
        D & 2 & 2\\
        E & 10 & 10\\
        F & 1 & 1\\
        G & 3 & 3\\
        H & 9 & 9\\
        I & 7 & 7\\
        J & 8 & 8\\
    \end{tabular}
\end{center}

    Begin traversing the tree backwards, comparing c values according to the algorithm. \\
    At E, c(I) is less than n(E). Therefore, update c(E), c(H), and c(J) (J and H can take forward edges to the back edge).\\
    \begin{center}
    \begin{tabular}{c | c c }
        & n & c\\
        \hline
        A & 5 & 5\\
        B & 4 & 4\\
        C & 6 & 6\\
        D & 2 & 2\\
        E & 10 & \textbf{7}\\
        F & 1 & 1\\
        G & 3 & 3\\
        H & 9 & \textbf{7}\\
        I & 7 & 7\\
        J & 8 & \textbf{7}\\
    \end{tabular}
\end{center}

    At I, c(A) is less than n(I).\\
    \begin{center}
    \begin{tabular}{c | c c }
        & n & c\\
        \hline
        A & 5 & 5\\
        B & 4 & 4\\
        C & 6 & \textbf{5}\\
        D & 2 & 2\\
        E & 10 & 7\\
        F & 1 & 1\\
        G & 3 & 3\\
        H & 9 & 7\\
        I & 7 & \textbf{5}\\
        J & 8 & 7\\    
    \end{tabular}
\end{center}

    At B, c(D) is less than n(B).\\ 
    \begin{center}
    \begin{tabular}{c | c c }
        & n & c\\
        \hline
        A & 5 & 5\\
        B & 4 & \textbf{2}\\
        C & 6 & 5\\
        D & 2 & 2\\
        E & 10 & 7\\
        F & 1 & 1\\
        G & 3 & \textbf{2}\\
        H & 9 & 7\\
        I & 7 & 5\\
        J & 8 & 7\\    
    \end{tabular}
\end{center}

    In our final table, nodes A, B, D, and I are not root nodes and have n values less than or equal to the c value of one of their children. 
    F, our root node, has only one child, so it is not an articulation point.
    The articulation points are A, B, D, and I.


\end{homeworkProblem}
\begin{homeworkProblem}
    Use induction on \(p\) to show that if \(G\) is a connected graph of order \(p\), then the size of \(G\) is at least \(p-1\).
    \\

    \textbf{Solution}

    \begin{proof}
        Base case: With \(p=1\), \(G\) is connected trivially with size 0. In this case, the size of \(G\) is at least \(p-1\).
        \\

        Inductive case: Assume that we have a connected graph \(G\) of order \(p\) with size \(s_0 \geq p-1\). 
        If we add another vertex to this graph and want to connect it to the graph, we must also add at least one edge.
        Therefore, a new graph with an added connected vertex will have size \(s_1 \geq s_0 + 1 \geq p-1+1\).
        This new graph of order \(p+1\) therefore also satisfies the size condition.
    \end{proof}
\end{homeworkProblem}

\begin{homeworkProblem}
    Find the smallest 3-regular graph simple graph having \(k(G)=1\).
    \\

    \textbf{Solution}
    \\

    Through experimentation, I have found that the smallest 3-ragular simple graph with a vertex cut set size of 1 is a bridged graph with 10 vertices like the following.\\
 
    \tikzset{vertex/.style={circle,draw,minimum size=0.25cm,inner sep=1pt}}
    \begin{tikzpicture}
        \node[vertex] (a) {};
        \node[vertex] (b) [right= 1cm of a] {};
        \node[vertex] (c) [above= 1cm of b] {};
        \node[vertex] (d) [below= 1cm of b] {};
        \node[vertex] (e) [right= 1cm of b] {};

        \node[vertex] (f) [right= 1cm of e] {};
        \node[vertex] (g) [right= 1cm of f] {};
        \node[vertex] (h) [above= 1cm of g] {};
        \node[vertex] (i) [below= 1cm of g] {};
        \node[vertex] (j) [right= 1cm of g] {};

        \path[draw, thick]
        (a) edge node {} (b)
        (a) edge node {} (c)
        (a) edge node {} (d)
        (b) edge node {} (c)
        (b) edge node {} (d)
        (c) edge node {} (e)
        (d) edge node {} (e)

        (e) edge node {} (f)

        (f) edge node {} (h)
        (f) edge node {} (i)
        (g) edge node {} (h)
        (g) edge node {} (i)
        (g) edge node {} (j)
        (h) edge node {} (j)
        (i) edge node {} (j)
        ;
    \end{tikzpicture}
    \\
 
%   I am not sure how to rigorously prove this concept, but here is my approach:\\
%   The smallest 3-regular graph is a simple triangle:\\
%   \tikzset{vertex/.style={circle,draw,minimum size=0.25cm,inner sep=1pt}}
%   \begin{tikzpicture}[node distance = 1cm]
%       \node[vertex] (a) {};
%       \node[vertex] (b) [right of=a] {};
%       \node[vertex] (c) [above of=b] {};

%       \path[draw, thick]
%       (a) edge node {} (b)
%       (a) edge node {} (c)
%       (b) edge node {} (c);
%   \end{tikzpicture}
%   \\


\end{homeworkProblem}

\begin{homeworkProblem}
    Prove or disprove: If \(G\) is 2-connected, then for an \(arbitrary\ u - v\) path \(P\), there is \(some\ other\ u - v\) path edge-disjoint from \(P\).
    \\

    \textbf{Solution}

    \begin{proof}
    If a graph is 2-connected, there does not exist any cut vertex that would disconnect the graph upon removal.
    By Menger's theorem, a connected graph with no cut vertex also must have at least two internally disjoint paths between any two vertices.
%    In other words, \(G\) has no articulation points. 
%    If \(G\) has no articulation points, there is no point that exists in every path between two subgraphs of \(G\).
%    Therefore, every path in \(G\) has an alternative vertex-disjoint path.

    Path edges are defined by the vertices at each edge's endpoints. 
    Therefore, because every path in \(G\) has an alternative internally disjoint path, it must also have an alternative edge-disjoint path.
    \end{proof}
\end{homeworkProblem}
\end{document}
